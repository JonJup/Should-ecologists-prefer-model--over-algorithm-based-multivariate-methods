% !TeX spellcheck = en_US

\section{Discussion}
%------------------------------%
%& 		Repeat what I did		
%------------------------------%
	We analyzed 180 simulated abundance data sets, that differed in response types and sample sizes with four different statistical methods, to assess the methods performance when used to differentiate between causal and noise variables. 
	
%------------------------------%
%& 	one sentence conclusions	
%------------------------------%

    MvGLM and dbRDA performed best showing low FPRs and FNRs for all response combinations and sample sizes. 
	%
    CQO assigned high \textit{p}-values to noise variables resulting in FPRs lower than those of dbRDA but higher than MvGLM's. 
    %
    However, it had the highest FNR for the lower three significance levels, resulting largely from the high \textit{p}-values of linear variables. 
	%
    CCA also assigned high \textit{p}-values to linear variables and additionally assigned low \textit{p}-values to noise variables. 
    %
    The method performed worst in this evaluation, showing the highest FPR at all significance levels and the highest FNR at the two highest significance levels. \\
	
%------------------------------%
%& 			GLMmv				
%------------------------------%
    
    MvGLMs had the lowest FPR of all methods. 
    %
    The three noise variable \textit{p}-values that fell below 0.05 all occurred in \textit{LL} models, which violated the assumption of random residuals and thus would likely be identified as unreliable models.
	%
	The FNR was also low and resulted only through communities with the smallest sample size.
	%
	A drawback of MvGLMs is the long run time due to resampling \citep{Wang2012}.
	%
	The resampling is necessary because it ensures that inferences are valid, even if species are intercorrelated. 
	%
	Resampling observations across independent sites (i.e. rows) accounts for their possible correlation \citep{anderson2001new}. 
    %
	Models that explicitly consider correlation structure avoid resampling and can reduce computation time.
	%
	Such models have been proposed, e.g. by \citet{Jamil2012} who used the site effect of a Generalized Linear Mixed Model to induce equal correlation between all species pairs.
    %
    A clear drawback of this method is, however, that equal correlation between all species is as (im)plausible as no correlation. 
    %
    Structuring the residual covariance matrix is important as the number of parameters that need to be estimated rises quickly (e.g. 55 in the covariance matrix for 10 species).
    %
    MvGLMs can use an unstructured correlation matrix, but this is only advisable for data sets with many more sites than species and is computationally expensive. 
    %
    Another option is shrinking the correlation matrix towards identity using ridge regularization \citep{warton2008penalized, Warton2011a}.
	%
	Both alternatives use Generalized Estimation Equations (GEE) with the sandwich-type-estimator of \citet{Warton2011a}.
	%
	As GEEs do not provide likelihoods, other test statistics than the Likelihood ratio have to be used. 
	%
	Current options are the score and the Wald statistic.
	%
	However, these methods also require resampling, as asymptotic marginal distributions of regression parameters for GEEs are not specified for data sets with more species than sites.
	%
    Testing these methods on data sets with known correlation structures could highlight stronger performance differences, as the other methods lack adjustments to these properties.
    %
	MvGLMs are the only method considered here that does not reduce the dimensions of the data.
	%
	Visualizing the multidimensional data is difficult as no easy-to-use and interpret method for MvGLMs is available. \\

%------------------------------%
%& 		dbRDA		
%------------------------------%
    dbRDA was least influenced by different response types and sample sizes. 
    %
    These properties, together with the low FPR and FNR and the modest computation time make it an excellent method for cultivate analysis in ecology.
	%
	However, the FPR was higher than that of both model-based methods.
	%
	Small \textit{p}-values were scarce for noise variables but occurred at all samples sizes and response types. 
	%  
	dbRDA's good performance is in concert with other simulation studies \citep[e.g.][]{Roberts2009}.
	%
	These results are only valid for the Bray-Curtis distance metric, which was used here.
	%  
	Other measures would likely produce different results, 
	therefore the selection of an appropriate metric is a crucial step in any dbRDA analysis.
	%  
	Having to choose a single metric can be avoided by using consensus RDA \citep{Blanchet2014}.
	% 
	In this method, multiple dbRDAs are run, only differing in their distance metric. 
	%
	Site scores on statistically significant axes are combined into one matrix, which acts as a response matrix in a new RDA. 
	%
	This method extracts the information that is common to all individual dbRDAs.
	%
	Simulation studies comparing properties of consensus RDA with those of individual dbRDA and other methods, algorithm- or model-based, are lacking.
	% 
	Another avenue for the future development of distance-based algorithms, in general, would be novel distance metrics, but their development is pending (M. J.Anderson, pers. comm.).\\

%------------------------------%
%& 				CCA				
%------------------------------%

	The CCA performed worst of the methods tested and assigned high \textit{p}-values to all linear variables.
	%
	As CCA assumes unimodal gradients, which are more frequent than linear ones in nature \citep{Oksanen2002}, this was expected.  
	%
	This study confirmed, that CCA should be avoided if exploratory analyses indicate linear relationships, which can occur if the sampled range of a gradient is short relative to the species' tolerance. 
	%
	Noise \textit{p}-values were lower than in other methods.
	% 
	Most of the low \textit{p}-values for noise variables occurred in communities with uni- or bimodal responses. 
	%
	This is surprising, given that \textit{UU} fits the expectations of the species packing model perfectly and bimodal models deviate only slightly.
	%
	Newer approaches to CCA that can correct for zero inflation \citep{Zhang2012} or non-linear relationships between predictor and response variable \citep{Makarenkov2002} are available but not widely used. 
	%
	Indeed, all of the methods we tested here can include quadratic terms which would most likely have resulted in better fitting models for unimodal and bimodal predictors. 
	%
	Their application is uncommon in CCA and RDA and could be the scope of a future studies.

%------------------------------%
%& 			CQO					
%------------------------------%

	Similar to the CCA, CQO assigned high \textit{p}-values to linear variables. 
	%
    It also assumes unimodal responses and the non-detection of causal linear gradients was expected. 
    %
    The \textit{p}-values for linear variables of CQO were markedly lower than in the CCA, however, the \textit{p-value} of the second variable in these models tends to increase. 
    %
    Overall, this resulted in a high FNR.
	%
	The FPR however, was lower than for both algorithm-based methods but slightly higher than for MvGLM. 
	%
	These results reflect the performance of CQO when combined with our approach to compute \textit{p}-values, which to our knowledge, is novel. 
	%
CQO has only rarely been used in ecological studies and mostly within fisheries research (e.g. \citet{Vilizzi2012}, \citet{Top2016} and \citet{Carosi2017}). 
	% 
	\citet{TerBraak2015} suggest that this is due to limitations on the number of species that can be included, a steep learning curve and numerical instability. 
	%
	This study confirmed that in its current state the method has issues with linear response types but can handle alteration of the symmetrical unimodal bell-shape. \\

% ---------------------------------------- %
%&		Andere Vergleichende Studien		
% ---------------------------------------- %
 
 	Our study is the first to directly compare the methods. 
 	%
 \citet{Warton2012} compared MvGLMs to CCA and RDA (not dbRDA).
 	%
 	They showed that only MvGLMs successfully differentiate between the location effect (difference in means) and dispersion effect (difference in variance). 
 	%
 	Comparative studies of multivariate methods, in general, are common. 
 	%
 	Especially ordination techniques like CCA and RDA were subject to extensive testing in the 1970s and 1980s \citep[e.g.][]{GauchJr.1972, GauchJr1977, Kenkel1986}. 
 	%
 	To our knowledge, \citet{Roberts2008} and \citet{Roberts2009} are the only studies that systematically compared dbRDA to other methods. 
 	%
 	Both compared dbRDA, CCA and Multidimensional Fuzzy Set Ordinations. 
 	%
 	\citet{Roberts2008} used simulated data sets to this end, whereas \citet{Roberts2009} used four different field data sets. 
 	%
 	Both studies concluded that dbRDA outperforms CCA, as it does in this study.
 	%
 	CQO is occasionally tested in comparisons of individual and community level species distribution models (e.g. \citet{Baselga2009} and \citet{Maguire2016}),
 	where they are an instance of the latter.
 	%
 	Generally, they exhibited a similar performance as classical models (e.g. GLMs or Regression Trees). \\

%------------------------------%
%& 		Simulated Data			
%------------------------------% 
    The realism of the simulated communities could be improved by using more complex response patterns like beta-functions \citep{austin1994determining}, which add asymmetries to bell-shaped curves.
	%
	However, in a study of \citet{Oksanen2002} only about 20\% of the responses were strongly skewed, whereas symmetric and bell-shaped responses were most common. 
	%
	Alternatively, asymmetry could be introduced through random terms added to abundances, environmental variables, or both. \citep[e.g.][]{McCune1997}
	%
	When correlated random terms are added to both, this would engender endogeneity (a non-zero covariance between the residuals and one or more explanatory variables). 
	%
	Simulations with induced endogeneity would be interesting as this phenomenon is underappreciated by ecologists \citep{armsworth2009contrasting, fox2015ecological}.
	%
	Observation and measurement are sources of errors in field data sets, and both can be represented in a model via binomial functions as in N-mixture models \citep{royle2004n}.
	%
    This would be interesting to examine the effects of regression dilution \citep{frost2000correcting, McInerny2011}.
 
%------------------------------------------------%
%& Andere Vielversprechende Model-based approaches
%------------------------------------------------%

	Our study shows that model-based multivariate inference can outperform more frequently used algorithm-based methods. 
	%
	The answer to our eponymous question is thus: Not categorically - decisions should be made on a case-by-case basis.
	%
	As model-based methods are still at an early stage, new developments and increases in computation speed can be expected.   
	%
	An especially active area are models using joint probability distributions \citep[e.g.][]{Clark2014, Pollock2014} that estimate the joint distribution of all species conditional on the environmental variables instead of only using the marginal distribution of every species' abundance. 
	%
	A common interest of many joint models is to infer biotic interactions from the residuals of the species-environment interaction, 
	as these two sets of predictors (biotic and abiotic) were shown to have little redundancy \citep{Meier2010}.
	%
	Some of the models also anticipate the growing challenges of Big Data for ecology \citep{Hampton2013}.
	%
	Generalized Linear Latent Variable Models, for example, include latent variables instead of random effects to capture residual correlation, which considerably reduces the size of the variance -- covariance matrix \citep{Warton2015,Niku2017}.  
	%
	In Hierarchical Modeling of Species Communities \citep{Ovaskainen2017} this approach is coupled with a fourth corner model \cite[including species traits, ][]{legendre1997relating} and phylogenetic relationships to create a flexible and comprehensive framework for community data analysis.
	%
	In a similar vein, Generalized Joint Attribute Models allow for different kinds of data (e.g. continuous, discrete counts, ordinal counts, and occurrence) to be included in the same response variable and have outperformed Poisson GLM on discrete count data and a Bernoulli GLM on binary host status data in a recent simulation study \citep{Clark2017}. 
	%
	Lastly, \citet{anderson2019pathway} recently highlighted a combination of the model- and algorithm-based approaches.
    %
    They proposed a copula model of ecological count data \citep[see][for an introduction to copula models]{hofert2018elements}, which consists of i) fitting a copula model to the data, ii) simulating new count data with this copula and iii) visualizing the centroids of the actual data and of the simulated data sets in a PCoA.
    %
    In light of the good performance of dbRDA in our study, this proposal, to join features from both approaches, should be further pursued. 
    %
	It is now essential that ways to infer ecological processes from the modeled patterns develop at a similar pace as these models, to avoid confusing statistical artifacts with genuine biological signals \citep{dormann2018biotic}.
	%
	If this succeeds, a move from algorithm-based towards model-based methods might entail one from the current implicit Gleassonian towards a modern form of Clementsian perspective \citep{Eliot2011}; from asking how do individual species change along environmental gradients towards asking how do communities change as a whole.   


%---
%& GJAM
%---

 
% GJAM (avialable in gjam R package )  allow for different kinds of response data to be modeled for the same species. This includes continuous (e.g. biomass or density), discrete counts, ordinal counts (none, some, many) and incidence (present, absent). They also include the relative measurement effort (e.g. search time for counts or core volume sediment cores), since higher effort is likely to lead to reduced variability. In a simulation study GJAM outperformed Poisson GLM on discrete count data and a Bernoulli GLM on binary host status data \citep{Clark2017}.  


%%% --- Allgemein--- %%%
% - usually gradients are not the measured variables 
% - biotic interaction, succseional or priority effects not considered , changes in species richness along gradients.
% - GLMmv nicht ausgereizt. 



